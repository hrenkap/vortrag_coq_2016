\documentclass[aspectratio=169]{beamer}
\usepackage[utf8]{inputenc}

\usetheme{Berlin}

\title{Korrektheit von Programmen beweisen mit Coq}
\institute{Linux Tag Tübingen 2016}
\author{Peter Hrenka}
\date{\today}

\begin{document}
\begin{frame}
\titlepage
\end{frame}
\section*{Übersicht}
\begin{frame}
  \tableofcontents
\end{frame}
\section{Einführung}
\begin{frame}
  \frametitle{Über mich}
  \begin{itemize}
    \item Linux seit 1995
    \item Studium Informatik und Mathematik in Tübingen
    \item Sofwareentwickler C++, OpenGL
  \end{itemize}
\end{frame}
\subsection{Softwarequalität}
\begin{frame}
  ``Jedes nicht-triviale Programm hat Fehler''\\
  \pause
  \vfill
  Warum?
  \pause
  \vfill
  \begin{enumerate}
    \item Fehlendes Verständnis des Problems
    \pause
    \item Fehlerhafte ``Spezifikation''
    \pause
    \item Implementierung nicht korrekt
  \end{enumerate}
\end{frame}
\subsection{Methoden}
\begin{frame}
  Wie kann man die Korrektheit (bzgl. der Spezifikation) sicherstellen?
  \begin{itemize}
  \item Open Source
  \item Pair Programming
  \item Code Reviews
  \item Bug Bounties
  \item Testen
  \item Statische Analyse \pause
  \end{itemize}
  \vfill
  Alles hilft ein wenig aber nicht $100\%$ig 
\end{frame}
\section{Coq}
\subsection{Geschichte}
\begin{frame}
  \begin{itemize}
  \item Beweisassistenzsystem
  \item entwickelt am INRIA
  \item implementiert in OCaml
  \item verwendet ``dependend types''
  \item interaktiv
  \end{itemize}
\end{frame}
\subsection{Verwendung}
\section{Demo}
\begin{frame}
  \begin{center}
    \huge{Demo}
  \end{center}
\end{frame}
\section{Zusammenfassung}
\begin{frame}
    \begin{itemize}
  \item Sätze sind Typen
  \item Programme sind Beweise
  \item Induktion geht nicht nur mit $n\in\mathbb{N}$
  \item Automatisierung hilft
  \end{itemize}
\end{frame}
\section{Ausblick}
\begin{frame}
\end{frame} % to enforce entries in the table of contents

\end{document}