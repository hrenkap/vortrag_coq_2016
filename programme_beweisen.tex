\documentclass[aspectratio=169]{beamer}
\usepackage[utf8]{inputenc}
\usepackage[german]{babel}
\usepackage{graphics}
\usepackage{amsmath}

\usetheme{Berlin}

\title{Korrektheit von Programmen beweisen mit Coq}
\institute{Linux Tag Tübingen 2016}
\author{Peter Hrenka}
\date{\today}

\begin{document}
\begin{frame}
\titlepage
\end{frame}
\section*{Übersicht}
\begin{frame}
  \tableofcontents
\end{frame}
\begin{frame}
  \frametitle{Über mich}
  \begin{itemize}
    \item Linux Anwender seit 1995
    \item Studium Informatik und Mathematik in Tübingen
    \item Sofwareentwickler \texttt{C++}, \texttt{python}, OpenGL
  \end{itemize}
\end{frame}
\section{Einführung}
\subsection{Softwarequalität}
\begin{frame}
  \begin{center}
    \includegraphics[width=9.2cm]{projekt-schaukel-baum.png}
  \end{center}
\end{frame}
\begin{frame}
  ``Jedes nicht-triviale Programm hat Fehler''\\
  \pause
  \vfill
  Warum?
  \pause
  \vfill
  \begin{enumerate}
    \item Fehlendes Verständnis des Problems
    \pause
    \item Fehlerhafte ``Spezifikation''
    \pause
    \item Implementierung nicht korrekt
  \end{enumerate}
\end{frame}
\subsection{Methoden}
\begin{frame}
  Wie kann man die Korrektheit (bzgl. der Spezifikation) sicherstellen?
  \begin{itemize}
  \item Testen
  \item Pair Programming
  \item Code Reviews
  \item Bug Bounties
  \item Open Source
  \item Statische Analyse \pause
  \end{itemize}
  \vfill
  Alles hilft ein wenig aber nicht $100\%$ig 
\end{frame}
\section{Coq}
\begin{frame}
  \begin{center}
    \includegraphics[width=1.0cm]{coq_logo.png}
    \Large{\texttt{Coq}}
  \end{center}
  \begin{itemize}
  \item Beweisassistenzsystem
  \item entwickelt am INRIA
  \item implementiert in \texttt{OCaml}
  \item verwendet ``dependend types''
  \item interaktiv
  \end{itemize}
\end{frame}
\begin{frame}
  \begin{center}
    \Large{Mathematik mit \texttt{Coq}}
  \end{center}
  \begin{itemize}
  \item Vier-Farben-Satz, 2005
  \item Satz von Feit-Thomson (Gruppentheorie), 2012
  \item Univalent Foundations, Homotopy Type Theory (HoTT), ca. 2012
  \end{itemize}
  \pause
  \vfill
  $\rightarrow$ scheint für Mathematik brauchbar, wie sieht es mit Informatik aus?
\end{frame}
\begin{frame}
  \begin{center}
    \Large{CompCert}
  \end{center}
  \begin{itemize}
  \item ``Compilers you can \textit{formally} trust''
  \item große Teilmenge von ISO C90 / ANSI C language
  \item MISRA-C 2004
  \item generiert effizienten Code für PowerPC, ARM and x86\\
    ``about 90\% of the performance of GCC version 4 at optimization level 1''
  \item implementiert in \texttt{OCaml} und \texttt{coq}
  \item geleitet von Xavier Leroy (LinuxThreads) 
  \item nicht-freie Lizenz, aber einige Teile unter GPL und BSD
  \end{itemize}
\end{frame}
\begin{frame}
  \begin{center}
    \includegraphics[width=20.0cm]{compcert_diagram.png}
  \end{center}
\end{frame}
\begin{frame}
  \begin{center}
    \includegraphics[width=8.0cm]{spark_logo.jpg}    
    %\Large{\texttt{SPARK}}
  \end{center}
  \begin{itemize}
  \item Erweitere Untermenge von \texttt{Ada}
  \item kann Vor- und Nachbedingungen teilweise zur Compilezeit prüfen
  \item kann \texttt{coq}-Code exportieren, um schwere Fälle manuell zu beweisen
  \item Pro- und GPL Editionen
  \end{itemize}
\end{frame}
\begin{frame}
  \begin{center}
    %\includegraphics[width=3.0cm]{why3_logo.png}    
    \Large{\texttt{Why3}}
  \end{center}
  \begin{itemize}
  \item Platform für Programmverfikation von Programmen in \texttt{WhyML}
  \item Anbindungen für andere Programmiersprachen existieren
    \begin{itemize}
    \item \texttt{Frama-C} für \texttt{C}
    \item \texttt{SPARK} für \texttt{Ada}
    \end{itemize}
  \item Automatische Solver: \texttt{Alr-Ergo}, \texttt{CVC3/4}, \texttt{Z3}, uvm. 
  \item Manuelle Solver: \texttt{coq}, \texttt{PVS} und \texttt{Isabelle/HOL}
  \end{itemize}
\end{frame}
\begin{frame}
  \begin{center}
    \includegraphics[width=3.0cm]{fscq_logo.png}
  \end{center}
  \begin{itemize}
  \item ``A Formally Certified Crash-proof File System''
  \item Nachweis, daß bei Absturz zu beliebigem Zeitpunkt keine Daten verloren gehen
  \item MIT 2015, u.A. Adam Chlipala
  \item implementiert in \texttt{coq}
  \item extrahierbar nach \texttt{ocaml}, \texttt{Haskell} oder \texttt{go}
  \item verwenbar mit \texttt{fuse} 
  \end{itemize}
\end{frame}
\section{Demo}
\begin{frame}
  \begin{center}
    \huge{Demo}
  \end{center}
\end{frame}
\section{Zusammenfassung}
\begin{frame}
  \begin{itemize}
  \item Sätze sind Typen
  \item Programme sind Beweise
  \item Induktion geht nicht nur mit $n\in\mathbb{N}$
  \item Automatisierung hilft
  \end{itemize}
\end{frame}
\section{Ausblick}
\begin{frame}
  \begin{center}
    \Large{Probleme}
  \end{center}
  \begin{itemize}
  \item Offizielle Dokumentation (höchstens) zum Nachschlagen geeignet
  \item Standardbibliothek mathematiklastig
  \item Unidirektione Arbeitsweise 
    \begin{itemize}
    \item \texttt{coq} $\longrightarrow$ \texttt{OCaml}
    \end{itemize}
  \item Kein Export nach \texttt{C}, \texttt{C++}
  \item 
  \end{itemize}

\end{frame} % to enforce entries in the table of contents

\end{document}